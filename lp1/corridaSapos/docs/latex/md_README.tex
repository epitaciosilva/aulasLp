\subsubsection*{Como executar}


\begin{DoxyItemize}
\item make
\item Insira as informações dos sapos e pistas nos arquivos pistas.\+txt e sapos.\+txt nos seguintes formatos\+:
\begin{DoxyItemize}
\item Sapos\+: nome-\/identificador
\item Pistas\+: nome-\/tamanho
\end{DoxyItemize}
\item ./bin/executavel
\end{DoxyItemize}

\subsubsection*{Descrição da tarefa}

Implemente em C++ um programa que simule a uma corrida de sapos. Implemente uma classe chamada Sapo contendo\+:
\begin{DoxyItemize}
\item Atributos\+: nome, identificador, distância percorrida, quantidade de pulos dados, quantidade de provas disputadas, vitórias, empates, quantidade total de pulos dados.
\item Atributo estático público\+: distância total da corrida
\item Métodos públicos\+:
\begin{DoxyItemize}
\item Construtor(es).
\item getters e setters, quando necessários.
\item pular\+:
\begin{DoxyItemize}
\item incrementa distância percorrida de forma randômica entre 1 e o máximo que o sapo pode saltar
\item Incrementa o número de pulos dados em uma unidade \subsubsection*{Especificação do Projeto}
\end{DoxyItemize}

O código desenvolvido deve seguir as especificações abaixo\+:
\end{DoxyItemize}
\end{DoxyItemize}
\begin{DoxyEnumerate}
\item O programa deverá ler um arquivo que conterá as informações dos sapos disponíveis para a corrida.
\item O programa deverá ler um arquivo que conterá as informações sobre as pistas disponíveis para a corrida.
\item Ao iniciar o programa, o usuário poderá\+: a) Ver estatísticas dos sapos. b) Ver estatísticas das pistas. c) Iniciar uma corrida. d) Criar sapos. e) Criar uma pista.
\item Para iniciar uma corrida\+: a) O usuário deverá escolher uma pista de corrida que os sapos irão disputar. b) É mostrado ao usuário a lista dos sapos que iram participar da corrida com seus nomes e números. c) O usuário dará o start (pei) da corrida.
\item Durante a corrida\+: a) Cada sapo irá pular individualmente, mostrando ao usuário seu nome, numeração e quanto ele pulou, em cada pulo. b) A medida que um sapo chegar na linha de chegada, ele não deverá mais pular nem emitir mais mensagens na tela. c) Quando o ultimo sapo terminar a corrida, o programa mostrará o Rank da corrida.
\item Lembre que as operações de criação de sapos e corridas devem salvar os mesmo nos arquivos, bem como as estatísticas dos sapos após as corridas realizadas.
\end{DoxyEnumerate}

\subsubsection*{Organização do projeto}

O código do projeto deve seguir a configuração de pastas e arquivos\+:
\begin{DoxyItemize}
\item /bin – código executável
\item /src – código fonte
\item /docs – documentação
\item makefile
\item R\+E\+A\+D\+ME – arquivo contendo informações sobre\+: configuração, compilação e execução. 
\end{DoxyItemize}